% !TeX spellcheck = sv_SE
\documentclass[a4paper, 12pt]{report}

\usepackage[swedish]{babel}
\usepackage[T1]{fontenc}
\usepackage{lmodern}
\usepackage[utf8]{inputenc}

\usepackage{amsmath}
\usepackage{amssymb}
\usepackage{amsfonts}
\usepackage{amsthm}

\usepackage{graphicx}

\DeclareMathOperator{\sign}{sign}




\title{Hilbertrum med Reproducerande Kärnor}
\author{Oscar Granlund}


\begin{document}
\maketitle

\begin{abstract}
	Testtesttesttesttesttest
\end{abstract}

\chapter{Stödvektormaskiner (SVM)}

\section{Klassificering med hjälp av separerande hyperplan}
Ponera problemet där man på basen av en mängd \textit{träningsdata} med paren $(\mathbf{x}_i,~y_i)$, $\mathbf{x}_i\in\mathbb{R}^p$ ,$y_i\in \{-1,~1\}$, försöker hitta en regel $g: \mathbb{R}^p \longmapsto \{-1,~1\}$ sådan att $g(\mathbf{x}_i)=y_i$ för alla träningspar $(\mathbf{x}_i,~y_i)$. Inom maskininlärningen och statistiken kallas liknande problem för klassificeringsproblem och många olika metoder används för att hitta lösningar. 

I detta kapitel behandlas en metod där plan med dimensionerna $p-1$ används för att definiera en regel som klassificerar \textit{observationerna} $\mathbf{x}_i$ i \textit{klasserna} $y_i\in\{-1,~1\}$ genom separering. Ett \textit{hyperplan} i ett vektorrum med dimensionen $p$ är ett underrum med dimensionen $p-1$; figur \ref{fig:separatinghyperplane} illustrerar ett separerande hyperplan för fallet $p=2$. Klassificeringsregeln $g$ för separerande hyperplan blir $g(\mathbf{x}_i)=\sign (\mathbf{x}_i^\intercal\boldsymbol{\beta} + \beta_0)$ där mängden $\{\mathbf{x}: \mathbf{x}^\intercal \boldsymbol{\beta} + \beta_0=0\}$, med $\mathbf{x},~\boldsymbol{\beta}\in \mathbb{R}^p$ och $\|\boldsymbol{\beta}\|=1$, definierar ett hyperplan, eller en \textit{affin} mängd, parametriserat av $\boldsymbol\beta$ och $\beta_0$.\cite{ESL}

\begin{theorem}
	Egenskaper hos hyperplan definierade som affina mängder $L=\{f(\mathbf{x})\}$.
\end{theorem} 

\begin{figure}
\centering
\includegraphics[width=0.8\linewidth]{example-image-a}
\caption{\label{fig:separatinghyperplane}20 datapunkter med ett separerande hyperplan (linje) där klassen $y=1$ har färgats blå och klassen $y=-1$ har färgats orange.}
\end{figure}
\bibliographystyle{plain}
\bibliography{bibliografi}
\end{document}          
